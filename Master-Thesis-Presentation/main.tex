\documentclass{beamer}

\usepackage[T1]{fontenc}
\usepackage[utf8]{inputenc}
\usepackage[english]{babel}
\usepackage{lmodern}

% Use Unipd as theme, with options:
% - pageofpages: define the separation symbol of the footer page of pages (e.g.: of, di, /, default: of)
% - logo: position another logo near the Unipd logo in the title page (e.g. department logo), passing the second logo path as option 
% Use the environment lastframe to add the endframe text
\usetheme[pageofpages=of]{Unipd}

\title{ \textbf{ \textit{SMKIT} } }
\subtitle{ \textbf{ \textit{Design and development of a social media kit} } }
\author[ \textbf{ \textit{Abdelilah Lahmer} } ]{ \textbf{ \textit{Lahmer Abdelilah} } }

\date{ \textbf{ \textit{December 12, 2024} } }

% The next block of commands puts the table of contents at the beginning of each section and highlights the current section
\AtBeginSection[]
{
  \begin{frame}
    \frametitle{Table of Contents}
    \tableofcontents[currentsection]
  \end{frame}
}



\begin{document}

% Make the title page
\frame{\titlepage}

% Insert the general toc
\begin{frame}{Table of Contents}
    \tableofcontents    
\end{frame}


\section{Introduction and Motivation}
    \begin{frame}{Background}
        \begin{itemize}
            \item Rapid growth of web content and social media.
            \item Challenges in content management and distribution.
        \end{itemize}
    \end{frame}

    \begin{frame}{Motivation}
        \begin{itemize}
            \item Need for automation in web content analysis and posting.
            \item Gaps in existing solutions.
        \end{itemize}
    \end{frame}


\section{Problem Statement}
    \begin{frame}{Problem Statement}
        \begin{itemize}
            \item Clear articulation of the problem SMKIT aims to solve.
            \item Emphasize the limitations of current tools.
            \item Importance of handling diverse content and metadata.
        \end{itemize}
    \end{frame}


\section{Objectives}
    \begin{frame}{Objectives}
        Outline thesis objectives:
        \begin{itemize}
            \item Automating content posting.
            \item Multi-platform support.
            \item Metadata extraction.
            \item Modularity and scalability.
        \end{itemize}
    \end{frame}


\section{Overview of SMKIT}
    \begin{frame}{High-Level Architecture}
        \begin{itemize}
            \item Show a diagram of SMKIT’s modular structure.
            \item Highlight generic and specialized modules.
        \end{itemize}
    \end{frame}

    \begin{frame}{Workflow}
        \begin{itemize}
            \item Illustrate the process: input (web pages) → analysis (metadata extraction) → output (social media/web posts).
        \end{itemize}
    \end{frame}

    \begin{frame}{Features}
        \begin{itemize}
            \item Metadata extraction (Open Graph tags, etc.).
            \item Multi-platform content posting (Twitter, Facebook, web).
            \item Customizable templates.
        \end{itemize}
    \end{frame}


\section{Technical Implementation}
    \begin{frame}{Core Components}
        \begin{itemize}
            \item Generic module for Open Graph metadata handling.
            \item Specialized Negapedia module.
        \end{itemize}
    \end{frame}

    \begin{frame}{Metadata Extraction}
        \begin{itemize}
            \item Discuss Open Graph tag analysis and fallback mechanisms.
        \end{itemize}
    \end{frame}

    \begin{frame}{Posting Process}
        \begin{itemize}
            \item Explain integration with social platforms APIs (Facebook, Twitter).
            \item Template usage for consistent identity.
        \end{itemize}
    \end{frame}

    \begin{frame}{Challenges}
        \begin{itemize}
            \item Discuss key technical challenges and solutions (e.g., handling metadata variability, scaling for multiple platforms).
        \end{itemize}
    \end{frame}


\section{Negapedia Module Case Study}
    \begin{frame}{Functionality}
        \begin{itemize}
            \item Summarization, comparison, and ranking modes.
            \item Unique features (conflict, anger levels, etc.).
        \end{itemize}
    \end{frame}

    \begin{frame}{Example Outputs}
        \begin{itemize}
            \item Show a few example posts generated from Negapedia data.
        \end{itemize}
    \end{frame}


\section{Results and Evaluation}
    \begin{frame}{Performance}
        \begin{itemize}
            \item Highlight results from testing (e.g., metadata extraction accuracy, processing time).
        \end{itemize}
    \end{frame}

    \begin{frame}{Effectiveness}
        \begin{itemize}
            \item Metrics like engagement improvements or user feedback.
        \end{itemize}
    \end{frame}

    \begin{frame}{Comparison with Existing Tools}
        \begin{itemize}
            \item Briefly discuss how SMKIT addresses gaps better than existing tools.
        \end{itemize}
    \end{frame}


\section{Contributions and Future Work}
    \begin{frame}{Contributions}
        \begin{itemize}
            \item Summarize key contributions of your thesis (e.g., modular framework, specialized Negapedia module, open-source tool).
        \end{itemize}
    \end{frame}

    \begin{frame}{Future Work}
        \begin{itemize}
            \item Discuss potential extensions (e.g., more modules, support for additional platforms).
        \end{itemize}
    \end{frame}


\section{Conclusion}
    \begin{frame}{Conclusion}
        \begin{itemize}
            \item Discuss potential extensions Recap the problem, solution (SMKIT), and key results.
            \item Closing statement on the significance of your work.
        \end{itemize}
    \end{frame}


\begin{emptyframe}
     \textbf{ \textit{Thank you!} }
\end{emptyframe}

\appendix
    
\end{document}