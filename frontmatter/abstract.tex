%!TEX root = ../dissertation.tex

This thesis presents a modular Social Media Kit (SMKIT\addtoglossary), a comprehensive framework designed to automate the analysis of web pages and the posting of content, including creating web pages and publishing on social media sites such as Facebook and Twitter. The SMKIT leverages a robust architecture to extract, analyze, and format metadata from diverse web content sources, using techniques such as Open Graph\footnote{Open Graph. URL: \url{https://ogp.me/}} tags analysis to gather structured data efficiently. 

The framework supports multiple input types, allowing the ingestion of web-based content directly through URLs, as well as local file system inputs, thus providing flexibility for different use cases and deployment environments. The SMKIT is architected to support extensibility through a modular design, enabling the creation of specialized modules tailored to the unique content characteristics and requirements of different websites.

For the purpose of this thesis, two specific modules were designed and developed: a generic module capable of processing a wide range of web pages by analyzing their OG\addtoglossary tags, and a specialized module tailored for the Negapedia\footnote{Negapedia. URL: \url{http://en.negapedia.org/}} website. The Negapedia-specific module includes custom data extraction and processing logic to handle the unique features of Negapedia content, such as conflict and polemic levels, as well as the significance of key terms.

The SMKIT employs customizable templates to format and post content in a manner consistent with the branding and style guidelines of different social media platforms, ensuring a seamless integration and uniform presentation across digital channels. By automating these processes, the SMKIT significantly reduces the time and effort required for content curation, while enhancing consistency, engagement, and reach on social media.

This thesis also details the implementation of the SMKIT, including the design choices, algorithmic strategies, and technical challenges encountered during its development. The results indicate that the SMKIT not only simplifies content management across multiple platforms but also offers a scalable solution for future expansions to include additional specialized modules for other websites or social media environments.