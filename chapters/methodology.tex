%!TEX root = ../dissertation.tex

\chapter{Methodology}
\label{chp:methodology}
This chapter outlines the methodology employed in the development and implementation of the Social Media Kit (SMKIT) project. It provides a comprehensive explanation of the tools, techniques, and processes used to design, build, and implement the system. The chapter begins with an overview of the system architecture, followed by detailed discussions on data collection, processing, and the integration of various components.
This chapter sets the foundation for understanding the practical implementation of SMKIT, which will be further detailed in the subsequent chapters.


\section{Introduction}
\label{sec:introduction}
This section introduces the methodology used for developing the Social Media Kit (SMKIT). It provides an overview of the key techniques and tools employed throughout the development process, providing the foundation for the work presented in this thesis. The methodology chapter serves as a detailed explanation of how the objectives of SMKIT were achieved, with a focus on the following key areas:

\begin{itemize}
    \item \textbf{System Design and Architecture}: A comprehensive description of the system architecture of SMKIT, including the design decisions made to ensure the system’s scalability and functionality.
    \item \textbf{Data Collection and Preparation}: An overview of the data extraction techniques used, including web scraping tools and methods for cleaning and transforming the data into usable formats.
    \item \textbf{Implementation Process}: A detailed explanation of the implementation phase, where the system was developed and integrated with components like Negapedia and social media platforms.
    \item \textbf{Challenges and Limitations}: A reflection on the obstacles faced during the project and the limitations of the system that are important to consider when interpreting the results.
\end{itemize}

Each of these areas will be explored in detail in the subsequent sections of this chapter. By providing a comprehensive explanation of the methodologies used, this chapter ensures that the development process of SMKIT is well-understood.

The following sections will build upon this introductory overview, examining each of the mentioned aspects in greater detail to provide a complete picture of the technical and conceptual approaches adopted in the development of SMKIT.

\section{System Design and Architecture}
\label{sec:system_design_architecture}
This section explains the design decisions made for SMKIT, including the system architecture, components, and how they interact.

\subsection{System Architecture Overview}
\label{subsec:system_architecture_overview}
Provide a high-level overview of the system architecture, including diagrams that illustrate how different components of SMKIT interact.

\subsection{Technological Stack}
\label{subsec:technological_stack}
Describe the technologies and tools used to implement SMKIT, including any frameworks or programming languages like Python, Flask, or others.

\subsection{Module Development for Negapedia Integration}
\label{subsec:module_development_negapedia}
Explain the development of the specific module created to integrate data from Negapedia, highlighting any key challenges or innovations.

\section{Data Collection and Preparation}
\label{sec:data_collection_preparation}
This section describes how data is gathered, processed, and prepared for use in the SMKIT system.

\subsection{Web Scraping Tools and Techniques}
\label{subsec:web_scraping_tools_techniques}
Discuss the web scraping tools and techniques employed, such as BeautifulSoup or Scrapy, and how they were used to extract data from web pages.

\subsection{Data Cleaning}
\label{subsec:data_cleaning}
Explain the methods used to clean and process the raw data, including handling missing values, removing duplicates, and correcting errors.

\subsection{Data Transformation}
\label{subsec:data_transformation}
Describe the process of transforming raw data into a usable format for the system, such as converting text into structured data or normalizing values.

\section{Implementation}
\label{sec:implementation}
This section details the implementation of SMKIT, including the development of key modules and the integration of components.

\subsection{Module for Content Automation}
\label{subsec:module_for_content_automation}
Explain how the module for content automation was developed, including the algorithms or logic used to generate posts or content for social media platforms.

\subsection{Integration with Negapedia}
\label{subsec:integration_with_negapedia}
Describe how Negapedia’s data is integrated into the SMKIT system, focusing on the specific integration methods or APIs used.

\subsection{Social Media Platform Integration}
\label{subsec:social_media_integration}
Discuss the methods used to integrate SMKIT with social media platforms (e.g., Twitter, Facebook, Instagram) for automatic content sharing.

\section{Challenges and Limitations}
\label{sec:challenges_limitations}
This section discusses any technical, logistical, or methodological challenges encountered during the project, as well as the limitations of SMKIT.

\subsection{Technical Challenges}
\label{subsec:technical_challenges}
Describe the technical difficulties faced during development, such as integrating with social media APIs or dealing with large-scale data extraction.

\subsection{Limitations of SMKIT}
\label{subsec:limitations_of_smkit}
Explain the limitations of the SMKIT system, including any features or capabilities that are yet to be implemented or limitations in performance.

\section{Conclusion}
\label{sec:methodology_conclusion}
Summarize the key points of the methodology, reflect on the approach taken, and provide a brief transition to the next chapter, which will focus on the **Implementation** or **Results** of SMKIT.
