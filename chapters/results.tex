%!TEX root = ../dissertation.tex

\chapter{Results}
\label{chp:results}

This chapter presents the results of applying the Social Media Kit (SMKIT) in real-world scenarios. It evaluates the system's performance, assesses the impact of the generated social media posts, and analyzes the effectiveness of the system's components.

\section{Introduction}
\label{sec:results_introduction}
In this section, we introduce the purpose of the Results chapter and provide a high-level summary of the findings. We explain the goals of the evaluation and its significance in validating the performance, usability, and content quality of SMKIT.


\section{Performance Analysis}
\label{sec:performance_analysis}
This section discusses the overall performance of SMKIT, focusing on the following aspects:

\begin{itemize}
    \item \textbf{Execution Time}: The time taken to process a page, extract metadata, and post content to social media.
    \item \textbf{Efficiency}: How efficiently the system handles different amounts of data (e.g., number of pages or content types).
    \item \textbf{Data Throughput}: The rate at which SMKIT processes data, with an analysis of how well it handles various formats (e.g., `.gz` files and URLs).
\end{itemize}


\section{Accuracy of Metadata Extraction}
\label{sec:metadata_extraction_accuracy}
This section evaluates the accuracy of the metadata extraction performed by SMKIT. We measure the correctness of the extracted data, such as titles, descriptions, and images, and compare the automated extraction with manual content reviews.

\begin{itemize}
    \item \textbf{Comparison with Manual Extraction}: A comparison of the results produced by SMKIT versus manually verified data to ensure accuracy.
    \item \textbf{Handling Missing Data}: The system’s handling of missing metadata, including fallback mechanisms.
\end{itemize}


\section{Quality of Social Media Posts}
\label{sec:social_media_posts_quality}
In this section, we assess the quality of social media posts generated by SMKIT. We examine:

\begin{itemize}
    \item \textbf{Post Quality Assessment}: The quality of content generated for platforms such as Facebook, Twitter, and Web.
    \item \textbf{Engagement Metrics}: If available, we discuss the engagement with posts (likes, shares, comments).
    \item \textbf{Formatting Compliance}: How well SMKIT adheres to platform-specific guidelines (e.g., image sizes, text length limits).
\end{itemize}


\section{Effectiveness of Modules}
\label{sec:modules_effectiveness}
This section evaluates the effectiveness of the key modules in SMKIT:

\begin{itemize}
    \item \textbf{Generic Module}: Effectiveness in handling metadata extraction for general websites.
    \item \textbf{Negapedia Module}: The success of the Negapedia-specific module in generating summaries, comparisons, and rankings.
    \item \textbf{Social Media Platform Integration}: Evaluating the integration with platforms like Facebook and Twitter, considering API limits and connectivity.
    \item \textbf{Template Management}: How well SMKIT handles dynamic content generation across different templates and platforms.
\end{itemize}


\section{Usability and User Feedback}
\label{sec:usability_and_feedback}
This section discusses the usability of SMKIT, focusing on:

\begin{itemize}
    \item \textbf{User Experience}: Evaluates the ease of use, especially the command-line interface (CLI).
    \item \textbf{Feedback from Users}: Summarizes feedback from testers or real users about their experience with SMKIT.
\end{itemize}


\section{Limitations and Challenges}
\label{sec:limitations_and_challenges}
In this section, we acknowledge the limitations of the current implementation and the challenges encountered during testing.

\begin{itemize}
    \item \textbf{Known Issues}: Describes any technical or operational issues faced (e.g., rate limiting, platform-specific issues).
    \item \textbf{Limitations of the Current System}: Acknowledges areas where SMKIT could be further improved or optimized.
\end{itemize}


\section{Conclusion}
\label{sec:results_conclusion}
This section summarizes the key findings of the results chapter. It reflects on the performance, accuracy, and usability of SMKIT. Finally, we discuss the implications of these results for future development and improvements, providing a smooth transition into the next chapter on \textbf{Future Work}.
