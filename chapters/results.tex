%!TEX root = ../dissertation.tex

\chapter{Results}
\label{chp:results}
This chapter presents the results of applying the Social Media Kit (\textbf{SMKIT}) in real-world scenarios. It evaluates the system's performance, assesses the impact of the generated social media posts, and analyzes the effectiveness of the system's components.


\section{Introduction}
\label{sec:results_introduction}
This section introduces the Results chapter and provides a high-level overview of the findings. The goal is to summarize the key results of applying \textbf{SMKIT}, comparing expected outcomes with the actual performance.


\section{Requirements Satisfied}
\label{sec:requirements_satisfied}
This section discusses the aspects of the \textbf{SMKIT} that meet the initial requirements defined in the \textbf{Methodology} and \textbf{Specifications} chapters. It highlights features that work as expected and provide the intended functionality:
\begin{itemize}
    \item \textbf{Metadata Extraction}: Successfully extracting metadata from web pages.
    \item \textbf{Content Creation}: Generating posts for Facebook, Twitter, and web pages.
    \item \textbf{Negapedia Integration}: Posts for Negapedia content, including summary, comparison, and ranking posts.
    \item \textbf{Social Media Integration}: Seamless integration with Facebook and Twitter APIs.
\end{itemize}


\section{Requirements Not Satisfied}
\label{sec:requirements_not_satisfied}
This section identifies any aspects of the project that did not meet expectations or where the \textbf{SMKIT} did not fully satisfy the original requirements:
\begin{itemize}
    \item \textbf{Platform Limitations}: Issues with integration or limitations on certain social media platforms.
    \item \textbf{Incomplete Metadata}: Challenges with extracting specific metadata or handling incomplete data.
    \item \textbf{Formatting Issues}: Problems with platform-specific formatting or content layout.
\end{itemize}


\section{Limitations, Future Work, and Improvements}
\label{sec:limitations_future_work_improvements}
This section discusses the limitations of the current implementation and outlines potential future improvements:
\begin{itemize}
    \item \textbf{Limitations}: The CLI-based interface is not optimal for user experience; it’s harder for less technical users to interact with the tool.
    \item \textbf{Future Work}: 
    \begin{itemize}
        \item Adding functionality for calculating social post metrics (e.g., click-through rates, engagement).
        \item Implementing a \textbf{Graphical User Interface (GUI)} to improve the user experience.
        \item Expanding integration with additional social media platforms.
        \item Enhancing metadata extraction accuracy for more complex websites.
    \end{itemize}
    \item \textbf{Improvements}: Improving error handling, optimizing performance, and adding more customization options for posts.
\end{itemize}


\section{Social Media Post Examples and Web Pages}
\label{sec:social_media_post_examples}
In this section, we present screenshots from real posts generated by \textbf{SMKIT}, including:
\begin{itemize}
    \item \textbf{Generic Post}: A standard post created using the generic template (Facebook, Twitter, Web Page).
    \item \textbf{Negapedia Summary Post}: A post summarizing content from Negapedia (Facebook, Twitter, Web Page).
    \item \textbf{Negapedia Comparison Post}: A post comparing two or more Negapedia pages (Facebook, Twitter, Web Page).
    \item \textbf{Negapedia Ranking Post}: A ranking-style post based on Negapedia data (Facebook, Twitter, Web Page).
\end{itemize}
Examples of these posts and web pages are shown below:
% Include your screenshots of posts here, for example:
% \begin{figure}[h]
%    \centering
%    \includegraphics[width=0.8\textwidth]{path/to/facebook_post_screenshot.png}
%    \caption{Example of a Facebook post generated by SMKIT.}
%    \label{fig:facebook_post}
% \end{figure}


\section{Execution Timing Analysis}
\label{sec:execution_timing}
This section presents an analysis of the execution time for different post types and operations. We analyze the time taken for:
\begin{itemize}
    \item \textbf{Facebook Post Creation}: Time to generate and post content on Facebook.
    \item \textbf{Twitter Post Creation}: Time to generate and post content on Twitter.
    \item \textbf{Web Page Post Creation}: Time to generate and share web pages based on the same content.
    \item \textbf{Post Types}: Time comparisons for different types of posts, including generic, Negapedia summary, comparison, and ranking posts.
\end{itemize}


\section{Evaluation of Post Quality and Engagement}
\label{sec:evaluation_post_quality}
This section evaluates the quality and engagement of posts generated by \textbf{SMKIT}:
\begin{itemize}
    \item \textbf{Post Quality}: How well SMKIT adheres to platform-specific formatting requirements for Facebook, Twitter, and web pages.
    \item \textbf{Engagement Metrics}: Discusses how the posts performed in terms of likes, shares, comments, etc. (if available).
\end{itemize}


\section{User Experience (UX) Considerations}
\label{sec:user_experience}
This section reflects on the usability of \textbf{SMKIT}, particularly considering its current command-line interface (CLI):
\begin{itemize}
    \item \textbf{CLI Usability}: The challenges of using the tool through a command-line interface for non-technical users.
    \item \textbf{Suggestions for GUI}: Discusses the benefits of developing a graphical user interface (GUI) for enhanced user experience, accessibility, and ease of interaction.
\end{itemize}


\section{Conclusion}
\label{sec:results_conclusion}
This section summarizes the key findings of the Results chapter. It provides a final assessment of how well \textbf{SMKIT} met the project objectives and performs in real-world usage. The conclusion also provides a transition to the next chapter, offering a perspective on **Future Work** or **Conclusions**.
