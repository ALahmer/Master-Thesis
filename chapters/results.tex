%!TEX root = ../dissertation.tex

\chapter{Results}
\label{chp:results}
This chapter presents the results of employing the Social Media Kit (\textbf{SMKIT}) in real-world scenarios. The evaluation is structured to highlight both the strengths and the limitations of the system, providing a comprehensive view of its effectiveness in automating the process of generating and sharing social media content.


\section{Objectives Met}
\label{sec:objectives_met}
This section discusses the aspects of the \textbf{SMKIT} that successfully meet the objectives and goals. It highlights the features that have been successfully implemented and provide the intended functionality.

The following objectives have been successfully achieved by \textbf{SMKIT}:
\begin{itemize}
    \item \textbf{Metadata Extraction}: \textbf{SMKIT} successfully extracts metadata from web pages, including key Open Graph tags such as \texttt{og:title}, \texttt{og:description}, and \texttt{og:image}. This feature has been thoroughly tested across multiple websites, ensuring accurate extraction and processing of data required for social media posts.
    
    \item \textbf{Content Creation}: The system is capable of generating posts for various platforms, including Facebook, Twitter, and web pages. The posts are automatically formatted to meet platform-specific guidelines, ensuring consistency and correctness in terms of text length, image sizes, and other platform requirements.
    
    \item \textbf{Negapedia Integration}: \textbf{SMKIT} can generate posts specifically for Negapedia content. This includes generating summary posts, comparison posts, and ranking posts based on the data available in the Negapedia static pages. The system accurately processes Negapedia-specific data such as social interaction metrics and conflict-related data, ensuring high-quality posts for both social media platforms and web pages.
    
    \item \textbf{Social Media Integration}: The integration with the Facebook and Twitter APIs works seamlessly, enabling \textbf{SMKIT} to automatically post content to these platforms. The system correctly handles authentication and the posting process, ensuring that content is successfully shared with minimal manual intervention. The posts are also formatted to comply to each platform's specific content requirements.
\end{itemize}

In summary, \textbf{SMKIT} has successfully met the key objectives, demonstrating robust functionality in terms of metadata extraction, content generation for multiple platforms, and seamless integration with social media APIs.


\section{Improvement Areas}
\label{sec:improvement_areas}
This section identifies aspects of the \textbf{SMKIT} project that require improvement in order to create a more complete and efficient system. While \textbf{SMKIT} performs well in many areas, there are still challenges and limitations that need to be addressed. The following points highlight these areas for improvement:
\begin{itemize}
    \item \textbf{Platform Limitations}:
    Despite seamless integration with Facebook and Twitter, there were some limitations in specific platform functionalities, particularly with Twitter's occasional API post content check restrictions, which led to error response codes. Additionally, the system is currently only integrated with Facebook and Twitter, with no support for other popular platforms such as Instagram or LinkedIn. Expanding to these platforms could broaden the tool's reach and support a wider range of social media users.

    \item \textbf{Incomplete Metadata}:
    While \textbf{SMKIT} is successful in extracting the basic Open Graph metadata, such as title, description, and image, challenges remain in handling the full range of Open Graph tags. Currently, the system only extracts the basic tags required by the stakeholders. A potential improvement would be to expand the system's capabilities to handle all possible Open Graph tags, which would provide more comprehensive metadata for social media posts.

    \item \textbf{CLI Usability Improvement}: 
    The current command-line interface (CLI) is functional but not optimal for user experience. It requires technical knowledge to operate, making it challenging for less technically-inclined users. A significant improvement would be to develop a \textbf{Graphical User Interface (GUI)} to make the tool more user-friendly and accessible to a broader audience. This change would enhance the usability and overall experience of the system.

    \item \textbf{Social Post Metrics Calculation}: 
    Currently, \textbf{SMKIT} does not include functionality for calculating metrics related to social media posts, such as click-through rates, or engagement levels. Adding this functionality would provide valuable insights into the performance of posts, helping users evaluate the effectiveness of their content. This feature would significantly enhance the system's capabilities by allowing users to track and analyze the impact of their social media activity.
    
    \item \textbf{Error Handling Optimization}: There are opportunities to improve the system’s error handling. Currently, some error messages are not very informative, making it challenging to troubleshoot issues for less technically-inclined users. Enhancing the error handling system to provide more detailed and user-friendly error messages would improve the user experience.
\end{itemize}

In summary, while \textbf{SMKIT} performs well in many areas, there are still opportunities for improvement, particularly in platform integration, metadata extraction, and functionality expansion. Addressing these challenges—such as expanding platform support beyond Facebook and Twitter—will be essential for increasing the tool's reach and making it a more complete and efficient system in future versions.


\section{Social Media Post Examples and Web Pages}
\label{sec:social_media_post_examples}
In this section, we present screenshots from real posts generated by \textbf{SMKIT}, including:
\begin{itemize}
    \item \textbf{Generic Post}: A standard post created using the generic template (Facebook, Twitter, Web Page).
    \item \textbf{Negapedia Summary Post}: A post summarizing content from Negapedia (Facebook, Twitter, Web Page).
    \item \textbf{Negapedia Comparison Post}: A post comparing two or more Negapedia pages (Facebook, Twitter, Web Page).
    \item \textbf{Negapedia Ranking Post}: A ranking-style post based on Negapedia data (Facebook, Twitter, Web Page).
\end{itemize}
Examples of these posts and web pages are shown below:
% Include your screenshots of posts here, for example:
% \begin{figure}[h]
%    \centering
%    \includegraphics[width=0.8\textwidth]{path/to/facebook_post_screenshot.png}
%    \caption{Example of a Facebook post generated by SMKIT.}
%    \label{fig:facebook_post}
% \end{figure}


\section{Execution Timing Analysis}
\label{sec:execution_timing}
This section presents an analysis of the execution time for different post types and operations. We analyze the time taken for:
\begin{itemize}
    \item \textbf{Facebook Post Creation}: Time to generate and post content on Facebook.
    \item \textbf{Twitter Post Creation}: Time to generate and post content on Twitter.
    \item \textbf{Web Page Post Creation}: Time to generate and share web pages based on the same content.
    \item \textbf{Post Types}: Time comparisons for different types of posts, including generic, Negapedia summary, comparison, and ranking posts.
\end{itemize}


\section{Evaluation of Post Quality and Engagement}
\label{sec:evaluation_post_quality}
This section evaluates the quality and engagement of posts generated by \textbf{SMKIT}:
\begin{itemize}
    \item \textbf{Post Quality}: How well SMKIT adheres to platform-specific formatting requirements for Facebook, Twitter, and web pages.
    \item \textbf{Engagement Metrics}: Discusses how the posts performed in terms of likes, shares, comments, etc. (if available).
\end{itemize}


\section{User Experience (UX) Considerations}
\label{sec:user_experience}
This section reflects on the usability of \textbf{SMKIT}, particularly considering its current command-line interface (CLI):
\begin{itemize}
    \item \textbf{CLI Usability}: The challenges of using the tool through a command-line interface for non-technical users.
    \item \textbf{Suggestions for GUI}: Discusses the benefits of developing a graphical user interface (GUI) for enhanced user experience, accessibility, and ease of interaction.
\end{itemize}


\section{Conclusion}
\label{sec:results_conclusion}
This section summarizes the key findings of the Results chapter. It provides a final assessment of how well \textbf{SMKIT} met the project objectives and performs in real-world usage. The conclusion also provides a transition to the next chapter, offering a perspective on **Future Work** or **Conclusions**.
