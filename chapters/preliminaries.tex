%!TEX root = ../dissertation.tex

\chapter{Preliminaries}
\label{ch:preliminaries}
This chapter provides a comprehensive overview of the foundational concepts, technologies, and frameworks that are essential for understanding the design, development, and functionality of the Social Media Kit (SMKIT) project.
It includes a detailed exploration of web content analysis, social media automation, metadata extraction techniques like Open Graph, and the integration of these technologies with existing platforms.
The chapter also introduces Negapedia, a platform that plays a central role in the specific implementation and use cases of SMKIT.

\section{Introduction to Web Content Analysis}
\label{sec:introduction_to_web_content_analysis}

\subsection{Definition and Importance}
\label{subsec:definition_and_importance}
Define web content analysis and discuss its significance in understanding user behavior, improving user experience, and enhancing search engine visibility.

\subsection{Techniques of Web Content Analysis}
\label{subsec:techniques_of_web_content_analysis}
Discuss various techniques used in web content analysis, such as text mining, sentiment analysis, content categorization, and data extraction.

\subsection{Applications in Social Media and Marketing}
\label{subsec:applications_in_social_media_and_marketing}
Explore how web content analysis is used in social media management and digital marketing to tailor content to audience needs and optimize engagement.

\section{Social Media Automation}
\label{sec:social_media_automation}

\subsection{Overview of Social Media Automation}
\label{subsec:overview_of_social_media_automation}
Explain the concept of social media automation, including its purpose and benefits in terms of time management, consistency, and analytics.

\subsection{Tools and Platforms for Social Media Automation}
\label{subsec:tools_and_platforms_for_social_media_automation}
Discuss popular tools and platforms that enable social media automation, such as Hootsuite, Buffer, and Zapier.

\subsection{Automation Strategies and Best Practices}
\label{subsec:automation_strategies_and_best_practices}
Detail common strategies for automating social media tasks, including content scheduling, post frequency management, and automated responses.

\section{Metadata and Open Graph Protocol}
\label{sec:metadata_open_graph}

\subsection{What is Metadata?}
\label{subsec:what_is_metadata}
Define metadata in the context of web content, its types, and its importance for search engine optimization (SEO) and content categorization.

\subsection{Introduction to Open Graph Protocol}
\label{subsec:introduction_to_open_graph_protocol}
Describe the Open Graph protocol, its development by Facebook, and its role in enabling rich social sharing.

\subsection{Key Open Graph Meta Tags}
\label{subsec:key_open_graph_meta_tags}
Detail the primary Open Graph tags (\texttt{og:title}, \texttt{og:description}, \texttt{og:image}, \texttt{og:video}, \texttt{og:url}, etc.), their attributes, and how they influence social media presentation.

\subsection{Implementing Open Graph for Content Optimization}
\label{subsec:implementing_open_graph_for_content_optimization}
Discuss how to implement Open Graph meta tags in HTML, and the tools and methods to validate and debug Open Graph markup.

\section{Existing Frameworks and Technologies}
\label{sec:frameworks_technologies}

\subsection{Overview of Relevant Web Technologies}
\label{subsec:overview_of_relevant_web_technologies}
Discuss the foundational web technologies (HTML, CSS, JavaScript) that are relevant to metadata and content sharing.

\subsection{Frameworks for Social Media Integration}
\label{subsec:frameworks_for_social_media_integration}
Explore existing frameworks and libraries (like \texttt{facebook-sdk}, \texttt{tweepy}, etc.) used for integrating social media platforms with web content.

\subsection{Content Management Systems (CMS) and Automation}
\label{subsec:content_management_systems_cms_and_automation}
Discuss the role of CMSs (like WordPress, Joomla) in managing web content and integrating Open Graph and social media features.

\section{Introduction to Negapedia}
\label{sec:introduction_to_negapedia}

\subsection{What is Negapedia?}
\label{subsec:what_is_negapedia}
Provide an overview of Negapedia, its purpose, and its unique approach to analyzing controversial topics across different languages.

\subsection{Key Features and Functionality of Negapedia}
\label{subsec:key_features_and_functionality_of_negapedia}
Detail the primary features of Negapedia, including its use of conflict and polemic metrics, and how it analyzes and categorizes content.

\subsection{Negapedia's Role in the SMKIT Project}
\label{subsec:negapedia_s_role_in_the_smkit_project}
Explain why Negapedia was chosen as a case study or example in the SMKIT project and how its data and analytics are utilized for content automation.

\section{Technologies and Protocols for Data Extraction}
\label{sec:technologies_and_protocols_for_data_extraction}

\subsection{Web Scraping and Data Extraction Techniques}
\label{subsec:web_scraping_and_data_extraction_techniques}
Discuss the techniques and tools (like BeautifulSoup, Scrapy) used for extracting metadata and content from web pages.

\subsection{Ethical and Legal Considerations in Web Scraping}
\label{subsec:ethical_and_legal_considerations_in_web_scraping}
Explore the ethical and legal aspects of web scraping, such as adherence to \texttt{robots.txt} and compliance with data privacy regulations.

\subsection{Data Cleaning and Transformation}
\label{subsec:data_cleaning_and_transformation}
Explain the methods used to clean and transform extracted data for use in social media automation and web content optimization.

\section{Conclusion}
\label{sec:preliminaries_conclusion}
Summarize the key points covered in the chapter.
Highlight the importance of understanding these concepts for the development and use of SMKIT. Provide a brief transition to the next chapter, which will focus on the system design and architecture of SMKIT.
